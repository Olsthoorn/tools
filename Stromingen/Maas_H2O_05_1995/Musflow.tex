% Options for packages loaded elsewhere
\PassOptionsToPackage{unicode}{hyperref}
\PassOptionsToPackage{hyphens}{url}
%
\documentclass[
]{article}
\usepackage{amsmath,amssymb}
\usepackage{lmodern}
\usepackage{iftex}
\ifPDFTeX
  \usepackage[T1]{fontenc}
  \usepackage[utf8]{inputenc}
  \usepackage{textcomp} % provide euro and other symbols
\else % if luatex or xetex
  \usepackage{unicode-math}
  \defaultfontfeatures{Scale=MatchLowercase}
  \defaultfontfeatures[\rmfamily]{Ligatures=TeX,Scale=1}
\fi
% Use upquote if available, for straight quotes in verbatim environments
\IfFileExists{upquote.sty}{\usepackage{upquote}}{}
\IfFileExists{microtype.sty}{% use microtype if available
  \usepackage[]{microtype}
  \UseMicrotypeSet[protrusion]{basicmath} % disable protrusion for tt fonts
}{}
\makeatletter
\@ifundefined{KOMAClassName}{% if non-KOMA class
  \IfFileExists{parskip.sty}{%
    \usepackage{parskip}
  }{% else
    \setlength{\parindent}{0pt}
    \setlength{\parskip}{6pt plus 2pt minus 1pt}}
}{% if KOMA class
  \KOMAoptions{parskip=half}}
\makeatother
\usepackage{xcolor}
\IfFileExists{xurl.sty}{\usepackage{xurl}}{} % add URL line breaks if available
\IfFileExists{bookmark.sty}{\usepackage{bookmark}}{\usepackage{hyperref}}
\hypersetup{
  hidelinks,
  pdfcreator={LaTeX via pandoc}}
\urlstyle{same} % disable monospaced font for URLs
\setlength{\emergencystretch}{3em} % prevent overfull lines
\providecommand{\tightlist}{%
  \setlength{\itemsep}{0pt}\setlength{\parskip}{0pt}}
\setcounter{secnumdepth}{-\maxdimen} % remove section numbering
\ifLuaTeX
  \usepackage{selnolig}  % disable illegal ligatures
\fi

\author{}
\date{}

\begin{document}

\hypertarget{can-you-give-me-the-1d-verical-richards-equation-for-unsaturated-groundwater-flow}{%
\section{Can you give me the 1D verical Richards equation for
unsaturated groundwater
flow?}\label{can-you-give-me-the-1d-verical-richards-equation-for-unsaturated-groundwater-flow}}

Of course. The 1D vertical Richards equation for unsaturated groundwater
flow is a partial differential equation that describes the movement of
water in the unsaturated zone, or vadose zone, of the soil. It is a
fundamental equation in hydrogeology and soil physics.

The equation is given by:

\[\frac{\partial \theta}{\partial t} = \frac{\partial}{\partial z} \left[ K(\psi) \left( \frac{\partial \psi}{\partial z} + 1 \right) \right]\]

Where: * \(\theta\) is the volumetric water content (unitless,
m\(^3\)/m\(^3\)) * \(\psi\) is the pressure head or matric potential (m)
* \(t\) is time (s) * \(z\) is the vertical coordinate, positive upwards
(m) * \(K(\psi)\) is the hydraulic conductivity, which is a function of
the pressure head (m/s)

This is the \textbf{mixed form} of the Richards equation. It is called
this because it uses both the volumetric water content (\(\theta\)) and
the pressure head (\(\psi\)) as variables.

\textbf{Alternative Forms:}

The Richards equation can also be written in two other common forms:

\textbf{1. Head Form:}

By using the chain rule, we can express the change in volumetric water
content with respect to time in terms of the pressure head:

\(\frac{\partial \theta}{\partial t} = \frac{d\theta}{d\psi} \frac{\partial \psi}{\partial t} = C(\psi) \frac{\partial \psi}{\partial t}\)

where \(C(\psi) = \frac{d\theta}{d\psi}\) is the specific moisture
capacity (1/m).

Substituting this into the mixed form, we get the \textbf{head form} of
the Richards equation:

\[C(\psi) \frac{\partial \psi}{\partial t} = \frac{\partial}{\partial z} \left[ K(\psi) \left( \frac{\partial \psi}{\partial z} + 1 \right) \right]\]

\textbf{2. Flux Form:}

The term
\(q = -K(\psi) \left( \frac{\partial \psi}{\partial z} + 1 \right)\)
represents the water flux, where the positive z-direction is upward. The
flux form of the Richards equation is derived from the continuity
equation:

\[\frac{\partial \theta}{\partial t} = - \frac{\partial q}{\partial z}\]

This form is useful for understanding the conservation of mass.

\textbf{Key Concepts:}

\begin{itemize}
\tightlist
\item
  \textbf{Unsaturated Zone:} The region of the soil above the water
  table where the soil pores are only partially filled with water.
\item
  \textbf{Matric Potential (\(\psi\)):} The pressure head of the soil
  water, which is negative in the unsaturated zone due to capillary
  forces and adsorption.
\item
  \textbf{Hydraulic Conductivity (\(K(\psi)\)):} The ability of the soil
  to transmit water. In the unsaturated zone, it is a function of the
  matric potential, and it decreases as the soil becomes drier.
\end{itemize}

The Richards equation is highly non-linear due to the dependence of the
hydraulic conductivity on the pressure head, which makes it challenging
to solve analytically. Numerical methods are typically used to solve it.

\hypertarget{how-about-the-lienarized-richards-equations}{%
\section{How about the lienarized Richards
equations?}\label{how-about-the-lienarized-richards-equations}}

You're right to ask about the linearized Richards equation, as it's a
crucial step in making the equation solvable, both analytically and
numerically. The original Richards equation is highly non-linear because
both the hydraulic conductivity, K(ψ), and the specific moisture
capacity, C(ψ), are strongly dependent on the pressure head, ψ.

Linearization involves making simplifying assumptions to transform this
non-linear PDE into a linear one, which is much easier to solve. The
most common way to do this is to approximate the non-linear hydraulic
properties as constants or as linear functions.

Here is the 1D vertical linearized Richards equation in the head form:

\[C \frac{\partial \psi}{\partial t} = K \frac{\partial^2 \psi}{\partial z^2} + \frac{\partial K}{\partial z}\]

This looks like a standard diffusion-advection equation, which is much
simpler than the original.

\textbf{Assumptions and Approximations:}

To arrive at this linearized form, several key assumptions are made:

\begin{enumerate}
\def\labelenumi{\arabic{enumi}.}
\item
  \textbf{Constant Hydraulic Conductivity (\(K\)):} The most significant
  approximation is assuming that the hydraulic conductivity, \(K\), is a
  constant value. In reality, \(K\) can vary by several orders of
  magnitude as the soil moisture changes. This assumption is only valid
  for small changes in matric potential, where the change in \(K\) is
  negligible. For example, it might be a reasonable approximation for a
  very wet soil profile where the soil is nearly saturated.
\item
  \textbf{Constant Specific Moisture Capacity (\(C\)):} Similarly, the
  specific moisture capacity, \(C = d\theta/d\psi\), is assumed to be a
  constant. This also implies a linear relationship between volumetric
  water content (\(\theta\)) and pressure head (\(\psi\)), which is
  often a crude approximation of the actual soil water retention curve.
\item
  \textbf{Linearization of the Gravity Term:} The original Richards
  equation has a gravity term that is also a function of the hydraulic
  conductivity. The linearization process often simplifies this term as
  well.
\end{enumerate}

\textbf{Limitations of the Linearized Equation:}

The linearized Richards equation has a major drawback: it is only
accurate for a very narrow range of conditions. It is not suitable for
modeling situations with a wide range of moisture contents, such as
infiltration into dry soil, or evaporation from a wet soil. For these
cases, the non-linearities of the hydraulic properties are too
significant to be ignored.

\textbf{Applications:}

Despite its limitations, the linearized Richards equation is useful in
some cases, particularly for:

\begin{itemize}
\item
  \textbf{Analytical Solutions:} Because of its simplicity, it is
  possible to find analytical solutions to the linearized Richards
  equation for simple boundary and initial conditions. These solutions
  are valuable for understanding the fundamental behavior of unsaturated
  flow and for benchmarking numerical models.
\item
  \textbf{Initial Estimates:} The linearized equation can provide a good
  first-order approximation for more complex numerical models. The
  solution can be used as an initial guess for the more accurate,
  non-linear iterative solvers (like Newton or Picard methods) used in
  numerical simulations.
\end{itemize}

In summary, the linearized Richards equation is a simplified version of
the full non-linear equation, obtained by assuming constant hydraulic
conductivity and specific moisture capacity. While it's not a
general-purpose tool for modeling unsaturated flow, it is a valuable
theoretical and pedagogical tool, and it serves as a basis for more
complex numerical solutions.

\hypertarget{what-if-the-k-and-psi-are-linear-functions-of-theta-like-kk_0-k_1-theta---theta_avg-and-psi-psi_0-psi_1-theta---theta_avg}{%
\section{\texorpdfstring{What if the \(K\) and \(\psi\) are linear
functions of \(\theta\), like \(k=k_0 + k_1 (\theta - \theta_{avg})\)
and \(\psi = \psi_0 + \psi_1 (\theta - \theta_{avg})\)
?}{What if the K and \textbackslash psi are linear functions of \textbackslash theta, like k=k\_0 + k\_1 (\textbackslash theta - \textbackslash theta\_\{avg\}) and \textbackslash psi = \textbackslash psi\_0 + \textbackslash psi\_1 (\textbackslash theta - \textbackslash theta\_\{avg\}) ?}}\label{what-if-the-k-and-psi-are-linear-functions-of-theta-like-kk_0-k_1-theta---theta_avg-and-psi-psi_0-psi_1-theta---theta_avg}}

That's an excellent question that gets to the heart of how we approach
solving the Richards equation in practice. Your proposed relationships,
where hydraulic conductivity (\(K\)) and pressure head (\(\psi\)) are
linearized with respect to volumetric water content (\(\theta\)), are a
common and powerful approach. This is often referred to as the
\textbf{\(\theta\)-based form} of the Richards equation, and it can be a
more robust approach for numerical solutions than the head-based form.

Let's derive the linearized Richards equation using your linear
relationships:

\begin{itemize}
\item
  \(K(\theta) = K_0 + K_1 (\theta - \theta_{avg})\)
\item
  \(\psi(\theta) = \psi_0 + \psi_1 (\theta - \theta_{avg})\)
\end{itemize}

Where:

\begin{itemize}
\item
  \(K_0\) and \(\psi_0\) are the hydraulic conductivity and pressure
  head at the average water content, \(\theta_{avg}\).
\item
  \(K_1\) and \(\psi_1\) are constants representing the slope of the
  linear relationships.
\end{itemize}

We'll start with the mixed form of the Richards equation:

\[\frac{\partial \theta}{\partial t} = \frac{\partial}{\partial z} \left[ K(\psi) \left( \frac{\partial \psi}{\partial z} + 1 \right) \right]\]

Now, we need to express the right-hand side in terms of \(\theta\). We
can use the chain rule to relate the spatial derivatives:

\(\frac{\partial \psi}{\partial z} = \frac{d\psi}{d\theta} \frac{\partial \theta}{\partial z}\)

From your linear relationship, the derivative of \(\psi\) with respect
to \(\theta\) is simply:

\(\frac{d\psi}{d\theta} = \psi_1\)

So, we have:

\(\frac{\partial \psi}{\partial z} = \psi_1 \frac{\partial \theta}{\partial z}\)

Now, let's substitute this and your linear \(K(\theta)\) function into
the mixed form:

\[\frac{\partial \theta}{\partial t} = \frac{\partial}{\partial z} \left[ (K_0 + K_1 (\theta - \theta_{avg})) \left( \psi_1 \frac{\partial \theta}{\partial z} + 1 \right) \right]\]

Now, we expand the product inside the brackets:

\[\frac{\partial \theta}{\partial t} = \frac{\partial}{\partial z} \left[ (K_0 + K_1 \theta - K_1 \theta_{avg}) (\psi_1 \frac{\partial \theta}{\partial z} + 1) \right]\]

\[\frac{\partial \theta}{\partial t} = \frac{\partial}{\partial z} \left[ (K_0 + K_1 \theta - K_1 \theta_{avg}) \psi_1 \frac{\partial \theta}{\partial z} + (K_0 + K_1 \theta - K_1 \theta_{avg}) \right]\]

Now, we apply the product rule for differentiation with respect to \(z\)
to both terms:

\textbf{Term 1:}
\(\frac{\partial}{\partial z} \left[ (K_0 + K_1 \theta - K_1 \theta_{avg}) \psi_1 \frac{\partial \theta}{\partial z} \right]\)

Let's define the diffusivity
\(D(\theta) = K(\theta) \frac{d\psi}{d\theta} = (K_0 + K_1 \theta - K_1 \theta_{avg}) \psi_1\).

This term is
\(\frac{\partial}{\partial z} \left[ D(\theta) \frac{\partial \theta}{\partial z} \right]\).

Using the product rule, this becomes:

\(\frac{\partial D}{\partial z} \frac{\partial \theta}{\partial z} + D \frac{\partial^2 \theta}{\partial z^2}\)

And since \(D\) is a function of \(\theta\), we use the chain rule again
for the first part:

\(\frac{dD}{d\theta} \frac{\partial \theta}{\partial z} \frac{\partial \theta}{\partial z} + D \frac{\partial^2 \theta}{\partial z^2} = \frac{dD}{d\theta} \left( \frac{\partial \theta}{\partial z} \right)^2 + D \frac{\partial^2 \theta}{\partial z^2}\)

From our linear relationships, \(\frac{dD}{d\theta} = K_1 \psi_1\).

So, Term 1 is:
\(K_1 \psi_1 \left( \frac{\partial \theta}{\partial z} \right)^2 + (K_0 + K_1 \theta - K_1 \theta_{avg}) \psi_1 \frac{\partial^2 \theta}{\partial z^2}\)

\textbf{Term 2:}
\(\frac{\partial}{\partial z} \left[ K_0 + K_1 \theta - K_1 \theta_{avg} \right]\)

This is the gravity term. Since \(K_0\), \(K_1\), and \(\theta_{avg}\)
are constants, the derivative with respect to \(z\) is:

\(K_1 \frac{\partial \theta}{\partial z}\)

Combining both terms, the linearized Richards equation in the
\textbf{\(\theta\)-based form} is:

\[\frac{\partial \theta}{\partial t} = (K_0 + K_1 \theta - K_1 \theta_{avg}) \psi_1 \frac{\partial^2 \theta}{\partial z^2} + K_1 \psi_1 \left( \frac{\partial \theta}{\partial z} \right)^2 + K_1 \frac{\partial \theta}{\partial z}\]

This equation is a lot more complex than the simple linear form we saw
before, but it's still simpler than the full non-linear equation. It's a
\textbf{diffusion-convection-reaction} type equation where the
coefficients are now linear functions of \(\theta\).

\textbf{Further Linearization (for a simpler form):}

If we assume that the terms with the spatial derivatives of \(\theta\)
are small, we can simplify this even further. A common approach for this
is to neglect the squared term,
\(\left( \frac{\partial \theta}{\partial z} \right)^2\). However, a more
common and direct linearization is to assume that the coefficients are
constant, as in the previous answer. If we assume a constant
diffusivity, \(D = K_{avg} \psi_1\), and a constant conductivity,
\(K_{avg}\), we get the classical diffusion-advection equation:

\[\frac{\partial \theta}{\partial t} = D \frac{\partial^2 \theta}{\partial z^2} + \frac{dK}{d\theta} \frac{\partial \theta}{\partial z}\]

where \(\frac{dK}{d\theta} = K_1\) in your case. So, the simplified form
is:

\[\frac{\partial \theta}{\partial t} = (K_{avg} \psi_1) \frac{\partial^2 \theta}{\partial z^2} + K_1 \frac{\partial \theta}{\partial z}\]

\textbf{Why this approach is useful:}

\begin{itemize}
\item
  \textbf{Diffusivity:} The term
  \(D(\theta) = K(\theta) \frac{d\psi}{d\theta}\) is known as the
  \textbf{soil water diffusivity}. By expressing both \(K\) and \(\psi\)
  as functions of \(\theta\), you directly model the diffusivity, which
  can be a more physically intuitive approach.
\item
  \textbf{Mass Conservation:} The \(\theta\)-based form is often
  preferred in numerical modeling because it tends to be more
  \textbf{mass-conservative}. This is a critical property for numerical
  solvers, as it ensures that the total amount of water in the system is
  conserved over time.
\item
  \textbf{Analytical Solutions:} While still not as simple as the
  constant-coefficient linear equation, this form is more amenable to
  analytical solutions than the fully non-linear equation, especially
  for specific boundary conditions.
\end{itemize}

In summary, your linear relationships lead to a more advanced, but still
linearized, form of the Richards equation in terms of volumetric water
content (\(\theta\)). This form is a powerful tool for both analytical
solutions and stable numerical simulations.

\end{document}
