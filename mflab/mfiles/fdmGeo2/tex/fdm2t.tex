\begin{alltt}
\textcolor{keyword}{function} [Phi,Qin,Qs,Qx,Qy,Psi]=fdm2t(gr,t,Tx,Ty,Ss,IBOUND,HI,QI)
\textcolor{comment}{%FDM2 a 2D block-centred transient finite difference model}
\textcolor{comment}{% USAGE:}
\textcolor{comment}{%    [Phi,Qin,Qs,Qx,Qy,Psi]=fdm2(gr,t,Tx,Ty,Ss,IBOUND,HI,FQ)}
\textcolor{comment}{% Inputs:}
\textcolor{comment}{%    gr    = grid2DObj (see grid2DObj)}
\textcolor{comment}{%    Tx,Ty = transmissivities, either scalar or full 2D arrays}
\textcolor{comment}{%    IBOUND= boudary array as in MODFLOW (<0 fixed head, 0 inactive, >0 active)}
\textcolor{comment}{%    HI    = initial heads (STRTHD in MODFLOW)}
\textcolor{comment}{%    QI    = prescribed inflow for each cell.}
\textcolor{comment}{%    t     = times, dt = diff(t) will be the time steps}
\textcolor{comment}{% Outputs:}
\textcolor{comment}{%    Phi(Ny  ,Nx  ,Nt ) [ L  ]  Nt=numel(diff(t)); Ndt=numel(dt);}
\textcolor{comment}{%    Qin(Ny  ,Nx  ,Ndt) [L3/T]  flow into cells during time step}
\textcolor{comment}{%    Qx( Ny  ,Nx-1,Ndt) [L3/T]  horizontal cell face flow positive in positive xGr direction}
\textcolor{comment}{%    Qy( Ny-1,Nx  ,Ndt) [L3/T]  vertial    cell face flow, postive in positive yGr direction}
\textcolor{comment}{%    Psi(Ny+1,Nx-2,Ndt) [L3/T]  stream function (only useful if flow is divergence free)}
\textcolor{comment}{%    Qs( Ny  ,Nx  ,Ndt) [L3/T]  flow released from storage during the time step Dphi*S*V/Dt}
\textcolor{comment}{% TO 991017  TO 000530 001026 070414 090314 101130 140410}

theta = 0.67; \textcolor{comment}{% degree of implicitness  }
t     = permute(unique(t(:)),[3,2,1]); dt = diff(t,1,3); Ndt=numel(dt);
Nodes = reshape(1:gr.Nod,gr.size);               \textcolor{comment}{% Node numbering}
IE=Nodes(:,2:end);   IW=Nodes(:,1:end-1);
IS=Nodes(2:\textcolor{keyword}{end},:);   IN=Nodes(1:end-1,:);

Iact  = IBOUND(:) >0;
Inact = IBOUND(:)==0; Tx(Inact)=0; Ty(Inact)=0;
Ifh   = IBOUND(:) <0;

[Cx,Cy,Cs] = conductances(gr,Tx,Ty,Ss,theta);

C     = sparse([IW(:); IN(:)], [IE(:); IS(:)], [-Cx(:); -Cy(:)], gr.Nod, gr.Nod, 5*gr.Nod);
C     = C + C';
diagC = -sum(C,2);  \textcolor{comment}{% Main diagonal}

HI(Inact)=NaN;   Phi = bsxfun(@times, t,HI);   HI = HI(:); HT=HI;
QI(Inact)=  0;   Qin = bsxfun(@times,dt,QI);   QI = QI(:);

\textcolor{keyword}{for} idt = 1:Ndt
    HT(Iact) = spdiags(diagC(Iact)+Cs(Iact)/dt(idt),0,C( Iact , Iact )) \textcolor{keyword}{\underline{...}}
            \(\backslash\)(QI(Iact) - C(Iact,Ifh)*HI(Ifh) + Cs(Iact)/dt(idt).*HI(Iact)); \textcolor{comment}{% solve}
    HI(Iact) = HT(Iact)/theta - (1-theta)/theta * HI(Iact);
    
    Phi(:,:,idt+1)= reshape(HI,gr.size);
    \textcolor{keyword}{if} nargout>1
        Qin(:,:,idt)= reshape(spdiags(diagC(\~{}Inact),0,C(\~{}Inact,\~{}Inact))* HI(\~{}Inact),gr.size);    
    \textcolor{keyword}{end}
\textcolor{keyword}{end}

\textcolor{keyword}{if} nargout>2
    Cs =  reshape(Cs,gr.size);
    Qs =  bsxfun(@times,     dt, Cs) .* diff(Phi,1,3); \textcolor{comment}{% Release rate from storage}
    
    \textcolor{keyword}{if} nargout>3

        PhiTheta = @(theta) theta * Phi(:,:,2:end) + (1-theta) * Phi(:,:,1:end-1);

        onesDt = ones(size(dt));
        Qx = -bsxfun(@times, onesDt, Cx) .*diff(PhiTheta(theta),1,2); \textcolor{comment}{% Flow across vertical   cell faces}
        
        \textcolor{keyword}{if} nargout>4
            Qy = +bsxfun(@times, onesDt, Cy) .*diff(PhiTheta(theta),1,1); \textcolor{comment}{% Flow across horizontal cell faces}

            \textcolor{keyword}{if} nargout>5
                Psi = zeros(Ny+1,Nx-2,Ndt);
                Psi(2:\textcolor{keyword}{end},:,:) = cumsum(Qx,1);
                Psi(:,:,:) = Psi(end:-1:1,:,:);
            \textcolor{keyword}{end}
        \textcolor{keyword}{end}
    \textcolor{keyword}{end}
\textcolor{keyword}{end}

\textcolor{keyword}{function} [Cx,Cy,Cs] = conductances(gr,Tx,Ty,Ss,theta)
    \textcolor{comment}{%CONDUCTANCE --- compute conductances}
    \textcolor{comment}{% USAGE: [Cx,Cy] = contuctance(gr,Tx,Ty)}
    
    \textcolor{keyword}{if} isscalar(Tx), Tx= gr.const(Tx); \textcolor{keyword}{end}
    \textcolor{keyword}{if} isscalar(Ty), Ty= gr.const(Ty); \textcolor{keyword}{end}
    \textcolor{keyword}{if} isscalar(Ss), Ss= gr.const(Ss); \textcolor{keyword}{end}
    
    \textcolor{comment}{% resistances and conducctances}
    \textcolor{keyword}{if} \~{}gr.AXIAL
        RX = 0.5*bsxfun(@rdivide,gr.dx,gr.dy)./Tx;
        RY = 0.5*bsxfun(@rdivide,gr.dy,gr.dx)./Ty;
        Cx = 1./(RX(:,1:end-1)+RX(:,2:end));
        Cs = gr.Vol.*Ss/theta; Cs = Cs(:);
    \textcolor{keyword}{else}
        RX = bsxfun(@rdivide,log(gr.xGr(2:end-1)./gr.xm( 1:end-1)),2*pi*Tx(:,1:end-1).*gr.dY(:,1:end-1))+ \textcolor{keyword}{\underline{...}}
             bsxfun(@rdivide,log(gr.xm( 2:end  )./gr.xGr(2:end-1)),2*pi*Tx(:,2:end  ).*gr.dY(:,2:end  ));
        RY = 0.5*bsxfun(@rdivide,gr.dY./Ty, pi*(gr.xGr(2:end).\^{}2 - gr.xGr(1:end-1).\^{}2));
        Cs = gr.Vol.*Ss/theta; Cs = Cs(:);
        Cx = 1./RX;
    \textcolor{keyword}{end}
    Cy = 1./(RY(1:end-1,:)+RY(2:\textcolor{keyword}{end},:));

\end{alltt}
